\documentclass[a4paper]{article} %Always requireed
%========================================================================
% preamble to declare kackges used in the document
%========================================================================
%Packages that will be used
\usepackage{ragged2e}
\usepackage{tabularx}
\usepackage{multirow}
\usepackage[table,xcdraw]{xcolor}

%Define new command used for "tabularx" package
\newcolumntype{Y}{>{\centering\arraybackslash}X} %Define new column type (Centering with textwidth)
\renewcommand{\arraystretch}{1.4} %Increase the table height
%Set up caption formatting/alignment
\usepackage{caption}
\captionsetup {	
	%justification   = centering 
	justification   = raggedright,
	singlelinecheck = false
}
%\renewcommand{\arraystretch}{1.4}
%To use a block comment in a document
\usepackage{comment} 
\begin{comment}
	Test a block comment 
\end{comment}
%------------------------------------------------------------------------
\usepackage{fontspec}
%Either Xelatex or Lualatex is required as a default complier
%To learn more about "fontspec" package: https://ctan.org/pkg/fontspec?lang=en 
%========================================================================
%To declare a document title and an author(s) 
\title{EGCI491 Computer Engineering Seminar\\ \LaTeX\quad Assignment II}
\author{Firstname Lastname}

\begin{comment}
	\author{
		LastName1, FirstName1\\
		\texttt{first1.last1@xxxxx.com} 
		\and
		LastName2, FirstName2\\
		\texttt{first2.last2@xxxxx.com}
		\and
		LastName3, FirstName3\\
		\texttt{first3.last3@xxxxx.com}
		\and
		LastName4, FirstName4\\
		\texttt{first4.last4@xxxxx.com}
	}
\end{comment}
%document
\begin{document}
	\maketitle
	%Put your text in a body of your document here... 
	
	\section{Timeline}
	\newcolumntype{L}[1]{>{\raggedright\let\newline\\\arraybackslash\hspace{0pt}}m{#1}}
	\newcolumntype{C}[1]{>{\centering\let\newline\\\arraybackslash\hspace{0pt}}m{#1}}
	\newcolumntype{R}[1]{>{\raggedleft\let\newline\\\arraybackslash\hspace{0pt}}m{#1}}	
	\begin{table}[!ht]
		\footnotesize
		\sloppy
		\centering
		\caption{Project Timeline}
		\label{tab: your-table} %for cross-reference
		\begin{tabular}{|p{2.15cm}|c|c|c|c|c|c|c|c|c|c|c|c|}
			\hline
			\multicolumn{1}{|c|}{}& \multicolumn{12}{c|}{\textbf{Timeline}} \\ \cline{2-13} 
			\multicolumn{1}{|c|}{}& \multicolumn{12}{c|}{\textbf{2025}} \\ \cline{2-13} 
			\multicolumn{1}{|c|}{\multirow{-3}{2cm}{\textbf{Plan}}} & \textbf{Jan} & \textbf{Feb} & \textbf{March} & \textbf{April} & \textbf{May} & \textbf{Jun} & \textbf{July} & \textbf{Aug} & \textbf{Nov} & \textbf{Oct} & \textbf{Nov} & \textbf{Dec} \\ \hline
			1: Study PQA & 
			\cellcolor[HTML]{000000} &
			\cellcolor[HTML]{000000}  & & & & & & & & & & \\ \hline
			2: Study FPGA & & &
			\cellcolor[HTML]{000000} & 
			\cellcolor[HTML]{000000} & & & & & & & &\\ \hline
			3: Chapter1 & & & &
			\cellcolor[HTML]{000000} &
			& & & & & & & \\ \hline
			4: Chapter2 & & & & & 
			\cellcolor[HTML]{000000} & & & & & & &\\ \hline
			5: Chapter3 & & & & & &
			\cellcolor[HTML]{000000} &
			\cellcolor[HTML]{000000} & & & & & \\ \hline
			6: Testing & & &  & & & & &
			\cellcolor[HTML]{000000} &
			\cellcolor[HTML]{000000} &
			\cellcolor[HTML]{000000} &
			\cellcolor[HTML]{000000} &\\ \hline
			7: Chapter4 & & & & & & & & & &
			\cellcolor[HTML]{000000} & 
			\cellcolor[HTML]{000000} & \\ \hline
			8: Chapter5 & & & & & & & & & & & & 
			\cellcolor[HTML]{000000} \\ \hline
			9: Project Presentation& & & & & & & & & & & & 
			\cellcolor[HTML]{000000} \\ \hline
		\end{tabular}
	\end{table}
\end{document}
%========================================================================