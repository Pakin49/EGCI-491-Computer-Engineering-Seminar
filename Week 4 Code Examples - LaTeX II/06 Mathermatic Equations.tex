\documentclass[a4paper, 12pt]{article} %Always requireed, either Xelatex or Lualatex is required as a default complier
%======================================================
% preamble to declare kackges used in the document
%------------------------------------------------------
\usepackage[margin=1.5in]{geometry} %To adjust margins
\usepackage{comment} 	%To use a block comment in a document
%\usepackage{fontspec} 	%To learn more about "fontspec" package: https://ctan.org/pkg/fontspec?lang=en 
%---
\usepackage{amsmath} 	% Required to use the equation* environment

%To declare a document title and an author(s) 
\title{My First \LaTeX\ Document}
\author{Enter an author’s name here} %Replace your name here
\date{\today}

%======================================================
%document
\begin{document}
	\maketitle
	
	\setcounter{section}{6} %To start this section with 5
	\section{Mathematical Equations }
	\subsection{An Inline Formula}
	\noindent
	This is an example of the inline-formula typesetting $f(x) = x^2$ which is equivalent to \(f(x) = x^2\) and also to
	\begin{math}
		f(x) = x^2
	\end{math}.
	
	\subsection{A New-line Formulas without Number}
	This is a simple formula without numbering in a new line:\[f^2(x) = ax^2+bx+c\] which is equivalent to 
	\begin{displaymath} % a new-lne fomular without numbering
		f_{i=1}^{n}(x) = 2x_i^2+1
	\end{displaymath}
	% You can also use $$y = ax+b$$ (but not recommend as you may occasionally observe inconsistent vertical spacing
	
	\subsection{Multiple Lines of Formulas without Number and Alignment}
	The following is how to typeset multiple lines of formulas without number and alignment:
	\begin{align*} 
		ax^3+bx^2+c = 4y\\
		y = ax^3+bx^2+c \\
		3x+4 = 2y
	\end{align*}

	\subsection{Multiple Lines of Formulas without Number, but Alignment}
	The following is how to typeset multiple lines of formulas without number, but alignment:
	\begin{align*} 
		ax^3+bx^2+c &= 4y\\
		y &= ax^3+bx^2+c \\
		3x+4 &= 2y
	\end{align*}
	
	\subsection{Multiple Lines of Formulas with Number}
	The following is how to typeset multiple lines of formulas with number:
	\begin{align} %Typseting for multipl lines of wuations with numbering
		\begin{split} %Separate equations but count as one number
			ax^3+bx^2+c &= 4y\\
			&\quad +(2y^2+d)
		\end{split}\\
		y &= ax^3+bx^2+c\\
		3x+4 &= 2y
	\end{align}
\end{document}